\documentclass{article}
\usepackage[a4paper, margin=1in]{geometry}
\usepackage{enumitem}
\usepackage{hyperref}
\usepackage{setspace}

\pagenumbering{gobble}

\title{{\Huge Dados Projeto de}}
\author{}
\date{}
\begin{document}

% Capa
\setstretch{2}
\maketitle
\begin{center}
	{\LARGE Grupo 64}\\
	\vspace{8mm}
	{\LARGE Turno L08}\\
	\vspace{8mm}
	{\LARGE Prof. Miguel Garção Silva}\\

	\vspace{7cm}

	\begin{tabular}{|c|c|c|c|} \hline
		\textbf{Aluno}                    & \textbf{Esforço (horas)} & \textbf{Esforço Relativo} \\ \hline
		Bernardo Couto Melo (99057)       & 12 horas                 & 33\%                      \\ \hline
		Guilherme Marques Pascoal (99079) & 12 horas                 & 33\%                      \\ \hline
		Pedro Cerqueira Lobo (99115)      & 12 horas                 & 33\%                      \\ \hline
	\end{tabular}

\end{center}

\pagebreak

\setstretch{1.5}

\section{Desenvolvimento da Aplicação}
A aplicação foi desenvolvida utilizando \emph{HTML}, \emph{CSS}, \emph{Python} e o framework \emph{Flask}.\\
Todos os ficheiros necessários para o funcionamento da aplicação encontram-se
na pasta web. Os ficheiros \emph{HTML} estão na subdiretoria \texttt{templates}.
Existe também um ficheiro \emph{CSS} na subdiretoria \texttt{static} que visa tornar
mais apelativa a interface da aplicação.
Os detalhes da base de dados encontram-se no cabeçalho do ficheiro \texttt{app.cgi}.

A página inicial corresponde ao ficheiro \texttt{menu.html} apresenta ao
utilizador os submenus onde é possível realizar todas as operações exigidas.

O requisito a) corresponde às opções:
\begin{itemize}[leftmargin=+1.5cm]
	\setstretch{1}
	\item Adicionar Categoria
	\item Remover Categoria
	\item Adicionar Sub-Categoria
	\item Remover Sub-Categoria
\end{itemize}

O requisito b) corresponde às opções:
\begin{itemize}[leftmargin=+1.5cm]
	\setstretch{1}
	\item Adicionar Retalhistas
	\item Remover Retalhistas
\end{itemize}

O requisito c) corresponde à opção:
\begin{itemize}[leftmargin=+1.5cm]
	\setstretch{1}
	\item Listar Eventos de Reposição
\end{itemize}

O requisito d) corresponde à opção:
\begin{itemize}[leftmargin=+1.5cm]
	\setstretch{1}
	\item Listar Sub-Categorias
\end{itemize}

Ao realizar um operação com êxito não é apresentada nenhuma mensagem de sucesso
ao utilizador. Em vez disso, o utilizador é reencaminhado para o menu principal.
Se uma operação não for executada com êxito, é mostrada ao utilizador a mensagem
de erro emitida pelo SGBD.


\subsection{Menu Principal}
O utilizador é recebido com um menu principal, a partir do qual pode aceder a
todos os submenus, onde pode realizar todas as operações suportadas.

\subsection{Adicionar Categoria}
O utilizador é apresentado com uma caixa de texto, onde insere o nome categoria
a ser criada e um menu de opção, onde especifica se esta se trata de uma categoira
simples ou super categoria. Ao carregar no botão \texttt{Confirmar}, a categoria
é adicionada ao sistema.

\subsection{Remover Categoria}
É apresentada uma tabela com todas as categorias, com um botão \texttt{Remover},
junto a cada uma delas. Ao carregar nesse botão, o utilizador remove a categoria
do sistema.

\subsection{Adicionar Sub-Categoria}
São apresentadas duas tabelas. Na primeiram seleciona-se apenas uma das
sub-categorias e na segunda apenas uma das super-categorias. Ao carregar no botão
\texttt{Confirmar}, a sub-categoria selecionada é guardada como sub-categoria da
super-categoria selecionada.

\subsection{Remover Sub-Categoria}
É apresentada uma tabela com as várias relações \texttt{Super-Categoria/Sub-Categoria}.
Ao clicar no botão \texttt{Remover}, junto a cada uma destas relações, a cada uma das entradas da lista, a relação entre as duas categorias apresentadas é removida do sistema.

\subsection{Adicionar Retalhista}
São apresentadas uma caixa de input numérico, que recebe o TIN do retalhista,
uma caixa de texto para inserir o nome do mesmo e, por fim, uma tabela.
A tabela é originada através de um produto cartesiano de todas as IVM's,
ainda sem retalhista responsável, com todas as categorias registadas no sistema.
Isto permite uma mais fácil utilização da aplicação por parte do utilizador, pois
diminui os possíveis erros que este pode cometer ao introduzir os vários campos
requisitados. Contudo, incorretamente, o utilizador pode selecionar a mesma IVM,
para duas categorias diferentes. Para evitar isto, é apresentado ao utilizador um
aviso, imediatamente acima da tabela.
Selecionar uma ou mais entradas da tabela, equivale a tornar o retalhista
reponsável pela categoria selecionada, na IVM em questão.
Ao carregar no botão \texttt{Confirmar}, é adicionado um retalhista, com todas
essas responsabilidades.

\subsection{Remover Retalhista}
É apresentada uma lista com os vários retalhistas.
Ao clicar no botão \texttt{Remover}, junto a cada um deles, o retalhista é
removido do sistema.

\subsection{Listar Eventos de Reposição}
É apresentada uma lista com as várias IVM's. Ao clicar no botão \texttt{Listar},
junto a cada uma das IVM's, o utilizador é reencaminhado para uma página, onde
são listados todos os eventos de reposição que dizem respeito a essa IVM.

\subsection{Listar Sub-Categorias}
É apresentada uma lista com as várias super-categorias existentes no sistema.
Ao clicar no botão \texttt{Listar}, junto a cada uma delas, o utilizador é
reencaminhado para uma página onde são listadas todas as sub-categorias da
super-categoria selecionada, a todos os níveis de profundidade.

\subsection{Link}
\href{https://web2.ist.utl.pt/ist199115/app.cgi/}{https://web2.ist.utl.pt/ist199115/app.cgi/}


\pagebreak

\section{Índices}
\subsection*{Consulta 7.1}
\begin{verbatim}
	CREATE INDEX idx_tin_nome_cat ON responsavel_por USING HASH(tin, nome_cat);
	CREATE INDEX idx_nome ON retalhista(nome);
\end{verbatim}

\begin{itemize}
	\item As condições de filtragem sugerem a utilização de um índice em \texttt{tin}, para ambas as tabelas ou a utilização de um índice em \texttt{nome\_cat} para a tabela \texttt{responsavel\_por}. Obtámos por criar um índice de chave composta em \texttt{(tin, nome\_cat}) em \texttt{responsavel\_por}. Aproveitando o índice criado implicitamente na chave primária de \texttt{tin} em \texttt{retalhista}, a condição de filtragem será otimizada. O índice será não ordenado e de dispersão (\emph{HASH}), devido à igualdade da condição de filtragem.
	\item Poderia ser apenas utilizado um índice de dispersão apenas sobre o atributo \texttt{nome\_cat} ou apenas sobre o atributo \texttt{tin} de \texttt{responsavel\_por}, dependendo da seletividade de cada coluna.
	\item O uso de \texttt{DISTINCT} provoca uma ordenação, pelo que o uso de um índice de \texttt{nome} em \texttt{retalhista} pode acelerar a consulta. O índice será não ordenado e Btree.
\end{itemize}

\subsection*{Consulta 7.2}
\begin{verbatim}
	CREATE INDEX idx_desc ON produto(desc);
	CREATE INDEX idx_nome_ean ON tem_categoria(nome, ean);
\end{verbatim}

\begin{itemize}
	\item Poderá ser criado um índice de \texttt{desc} em \texttt{produto}, de modo a tirar partido de que a \texttt{desc} pretendida começa por `A'. O índice será não ordenado e Btree.
	\item A consulta beneficiaria também de um índice de \texttt{(nome, ean)} em \texttt{tem\_categoria}, de modo a otimizar a operação de agregação. O índice será não ordenado e Btree.
	\item É de notar que o índice criado implicitamente sobre \texttt{(ean, nome)} em \texttt{tem\_categoria} não traria o mesmo benefício que o índice anteriormente referido pois, neste \texttt{nome} surge depois de \texttt{ean}.
\end{itemize}

\end{document}
