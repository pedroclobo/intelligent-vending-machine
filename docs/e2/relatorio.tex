\documentclass{article}
\usepackage[a4paper, margin=1in]{geometry}
\usepackage{setspace}
\usepackage{enumitem}

\pagenumbering{gobble}

\newcommand{\select}{\sigma}
\newcommand{\project}{\Pi}
\newcommand{\njoin}{\bowtie}
\newcommand{\rename}{\rho}
\newcommand{\aggregatefn}{\mbox{\Large$G$}}


\title{{\Huge Projeto BD - Parte 2}}
\author{}
\date{}
\begin{document}

	% Capa
	\setstretch{2}
	\maketitle
	\begin{center}
		{\LARGE Grupo 64}\\
		\vspace{8mm}
		{\LARGE Turno L08}\\
		\vspace{8mm}
		{\LARGE Prof. Miguel Garção Silva}\\

		\vspace{7cm}

		\begin{tabular}{|c|c|c|c|} \hline
			\textbf{Aluno} & \textbf{Esforço (horas)} & \textbf{Esforço Relativo}\\ \hline
			Bernardo Couto Melo (99057)	& 4 horas & 33\%\\ \hline
			Guilherme Marques Pascoal (99079) & 4 horas & 33\%\\ \hline
			Pedro Cerqueira Lobo (99115) & 4 horas & 33\%\\ \hline
		\end{tabular}

	\end{center}

	\pagebreak

	% Modelo Relacional
	\section*{Modelo Relacional}
	\setstretch{1}
	\begin{itemize}

	 	\item[]{Product(\underline{ean}, descr)}

	 	\item[]{Category(\underline{name})}
		\begin{itemize}
			\item[$\bullet$]{IC-1: name tem de existir em Simple Category e/ou Super Category}
			\item[$\bullet$]{IC-2: name não pode existir em Simple Category e Super Category simultaneamente}
		\end{itemize}

	 	\item[]{Simple Category(\underline{name})}
		\begin{itemize}
			\item[$\bullet$]{name: FK(Category)}
		\end{itemize}

	 	\item[]{Super Category(\underline{name})}
		\begin{itemize}
			\item[$\bullet$]{name: FK(Category)}
		\end{itemize}

	 	\item[]{Retailer(\underline{TIN}, name)}
		\begin{itemize}
			\item[$\bullet$]{unique(name)}
		\end{itemize}

		\item[]{IVM(\underline{serial number}, \underline{manuf})}

		\item[]{Point of Retail()}

		\item[]{Replenishment Event()}

		\item[]{planogram()}

		\item[]{of()}

		\item[]{installed-at()}

		\item[]{responsible-for()}

		\item[]{replenisher-of()}

		\item[]{replenishment()}

		\item[]{has()}

		\item[]{displayed()}

		\item[]{has-other()}

	\end{itemize}

	\pagebreak

	% Álgebra Relacional
	\section*{Álgebra Relacional}
	\setstretch{1}

	\begin{enumerate}[label=(\arabic*)]
		\item{}
		\item{}
		\item{}
		\item{}
	\end{enumerate}

	\vspace{5mm}

	% SQL
	\section*{SQL}
	\setstretch{1}

	\begin{enumerate}[label=(\arabic*)]
		\item{}
		\item{}
		\item{}
		\item{}
	\end{enumerate}

\end{document}
