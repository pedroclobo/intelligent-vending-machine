\documentclass{article}
\usepackage[a4paper, margin=1in]{geometry}
\usepackage{setspace}
\usepackage{enumitem}

\pagenumbering{gobble}

\newcommand{\bpoint}{\item[$\bullet$]}

\newcommand{\select}{\sigma}
\newcommand{\project}{\Pi}
\newcommand{\njoin}{\bowtie}
\newcommand{\rename}{\rho}
\newcommand{\aggregatefn}{\mbox{$G$}}

\newenvironment{myitemize}
{ \begin{itemize}
	\setlength{\itemsep}{5pt}
	\setlength{\parskip}{0pt}
	\setlength{\parsep}{0pt} }
{ \end{itemize}	}


\title{{\Huge Projeto de BD - Parte 2}}
\author{}
\date{}
\begin{document}

	% Capa
	\setstretch{2}
	\maketitle
	\begin{center}
		{\LARGE Grupo 64}\\
		\vspace{8mm}
		{\LARGE Turno L08}\\
		\vspace{8mm}
		{\LARGE Prof. Miguel Garção Silva}\\

		\vspace{7cm}

		\begin{tabular}{|c|c|c|c|} \hline
			\textbf{Aluno} & \textbf{Esforço (horas)} & \textbf{Esforço Relativo}\\ \hline
			Bernardo Couto Melo (99057)	& 4 horas & 33\%\\ \hline
			Guilherme Marques Pascoal (99079) & 4 horas & 33\%\\ \hline
			Pedro Cerqueira Lobo (99115) & 4 horas & 33\%\\ \hline
		\end{tabular}

	\end{center}

	\pagebreak

	% Modelo Relacional
	\setstretch{1}
	\section*{Modelo Relacional}
	\vspace{2mm}
	\begin{myitemize}

		\item[] product(\underline{ean}, descr)
		\begin{myitemize}
				\bpoint RI-1: Todo o product (\textit{ean}) tem de participar na relação has
		\end{myitemize}

		\vspace{2mm}

		\item[] shelve(\underline{serial\_number}, \underline{manuf}, \underline{nr}, height, name)
		\begin{myitemize}
			\bpoint serial\_number, manuf: FK(ivm)
			\bpoint name: FK(category)
			\bpoint RI-2: (\textit{serial\_number}, \textit{manuf}, \textit{nr}) tem de existir em `ambient\_temp\_shelf', `warm\_shelf' ou `cold\_shelf'
			\bpoint RI-3: (\textit{serial\_number}, \textit{manuf}, \textit{nr}) não pode existir ao mesmo tempo em `ambient\_temp\_shelf', `warm\_shelf' ou `cold\_shelf'
		\end{myitemize}

		\vspace{2mm}

		\item[] ambient\_temp\_shelf(\underline{serial number}, \underline{manuf}, \underline{nr})
		\begin{myitemize}
			\bpoint (serial number, manuf, nr): FK(shelve)
		\end{myitemize}

		\vspace{2mm}

		\item[] warm\_shelf(\underline{serial number}, \underline{manuf}, \underline{nr})
		\begin{myitemize}
			\bpoint (serial number, manuf, nr): FK(shelve)
		\end{myitemize}

		\vspace{2mm}

		\item[]{cold\_shelf(\underline{serial number}, \underline{manuf}, \underline{nr})}
		\begin{myitemize}
			\bpoint (serial number, manuf, nr): FK(shelve)
		\end{myitemize}

		\vspace{2mm}

	 	\item[]{category(\underline{name})}
		\begin{myitemize}
			\bpoint RI-4: \textit{name} tem de existir em `simple\_category' ou `super\_category'
			\bpoint RI-5: \textit{name} não pode existir ao mesmo tempo em `simple\_category' e `super\_category'
		\end{myitemize}

		\vspace{2mm}

	 	\item[] simple\_category(\underline{name})
		\begin{myitemize}
			\bpoint name: FK(category)
		\end{myitemize}

		\vspace{2mm}

	 	\item[] super\_category(\underline{name})
		\begin{myitemize}
			\bpoint name: FK(Category)
			\bpoint RI-6: Toda a `super\_category' tem de participar na relação has-other
		\end{myitemize}

		\vspace{2mm}

	 	\item[] retailer(\underline{tin}, name)
		\begin{myitemize}
			\bpoint unique(name)
		\end{myitemize}

		\vspace{2mm}

		\item[] ivm(\underline{serial\_number}, \underline{manuf})

		\vspace{2mm}

		\item[] point\_of\_retail(\underline{address}, name)

		\vspace{2mm}

	\item[] replenishment\_event(\underline{ean}, \underline{serial\_number}, \underline{manuf}, \underline{nr}, \underline{instant}, tin, units)
		\begin{myitemize}
			\bpoint (ean, serial\_number, manuf, nr): FK(planogram)
			\bpoint tin: FK(retailer)
			\bpoint RI-7: units não pode exceder planogram.units
			\bpoint RI-8: has.name de ean tem de ser igual a planogram.shelve.name
			\bpoint RI-9: has.name de ean tem de ser igual a responsible-for.name de (serial\_number, manuf, tin)

		\end{myitemize}

		\pagebreak
		\vspace{2mm}

		\item[] planogram(\underline{ean}, \underline{serial\_number}, \underline{manuf}, \underline{nr}, faces, units, loc)
		\begin{myitemize}
			\bpoint ean: FK(product)
			\bpoint (serial\_number, manuf, nr): FK(shelve)
		\end{myitemize}

		\vspace{2mm}

		\item[] installed-at(\underline{serial\_number}, \underline{manuf}, address, nr)
		\begin{myitemize}
			\bpoint (serial\_number, manuf): FK(ivm)
			\bpoint address: FK(point\_of\_retail)
		\end{myitemize}

		\vspace{2mm}

		\item[] responsible-for(\underline{tin}, \underline{serial\_number}, \underline{manuf}, \underline{name})
		\begin{myitemize}
			\bpoint tin: FK(retailer)
			\bpoint (serial\_number, manuf): FK(ivm)
			\bpoint name: FK(category)
		\end{myitemize}

		\vspace{2mm}

		\item[] has(\underline{ean}, \underline{name})
		\begin{myitemize}
			\bpoint ean: FK(product)
			\bpoint name: FK(category)
		\end{myitemize}

		\vspace{2mm}

		\item[] has-other(\underline{category\_name}, super\_category\_name)
		\begin{myitemize}
			\bpoint category\_name: FK(category.name)
			\bpoint super\_category\_name: FK(super\_category.name)
			\bpoint RI-10: category\_name é sempre diferente de super\_category\_name
			\bpoint RI-11: super\_category\_name não pode ser category\_name, se category\_name for super\_category\_name, tendo também em conta os descendentes indiretos

		\end{myitemize}

		\vspace{5mm}

	\setstretch{1.5}

	\end{myitemize}

	\pagebreak

	% Álgebra Relacional
	\setstretch{1.5}
	\section*{Álgebra Relacional}

	\newcommand\responsiblefor{\mathop{\mbox{$responsible$-$\mathit{for}$}}}
	\newcommand\hasother{\mathop{\mbox{$has$-$\mathit{other}$}}}

	\begin{enumerate}[label=\arabic*)]

		\item $A \leftarrow \rename_{(2 \mapsto units)}(_{ean} \aggregatefn_{SUM(units)}(\select_{instant > ``2021/12/31"}(replenishment\_event)))\\
		       \project_{ean, desc}(\select_{units > 10\ \wedge\ name = ``Barras\ Energeticas"}(A \njoin product \njoin has))$

		\item $\project_{serial\_number}(\select_{ean = ``9002490100070"}(shelf \njoin has))$

		\item $\aggregatefn_{COUNT()}(\select_{super\_category\_name = ``Sopas\ Take-Away"}(\hasother))$

		\item $A \leftarrow \rename_{(2 \mapsto units)}(_{ean}\aggregatefn_{SUM(units)}(replenishment\_event))\\
		       B \leftarrow \rename_{(1 \mapsto max)}(\aggregatefn_{MAX(units)}(A))\\
		       \project_{ean, descr}(\select_{units=max}(product \njoin A \times B))$

	\end{enumerate}

	\vspace{5mm}

	% SQL
	\setstretch{1}
	\section*{SQL}

	\begin{enumerate}[label=\arabic*)]
		\item
			\begin{verbatim}
				SELECT ean, descr
				FROM product NATURAL JOIN has NATURAL JOIN (
				    SELECT ean
				    FROM replenishment_event
				    WHERE instant > `2021/12/31'
				    GROUP BY ean
				    HAVING SUM(units) > 10
				)
				WHERE name = `Barras Energéticas';
			\end{verbatim}

		\vspace{5mm}

		\item
			\begin{verbatim}
				SELECT serial_number
				FROM shelf NATURAL JOIN has
				WHERE ean = `9002490100070';
			\end{verbatim}

		\vspace{5mm}

		\item
			\begin{verbatim}
				SELECT COUNT(*)
				FROM has-other
				WHERE super_category_name = `Sopas Take-Away';
			\end{verbatim}

		\vspace{5mm}

		\item
			\begin{verbatim}
				SELECT ean, descr
				FROM replenishment_event NATURAL JOIN product
				GROUP BY ean
				HAVING SUM(units) >= ALL (
				    SELECT SUM(units)
				    FROM replenishment_event
				    GROUP BY ean
				);
			\end{verbatim}
	\end{enumerate}

\end{document}
